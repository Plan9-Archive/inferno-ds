\documentclass[11pt]{p9article}
\usepackage[T1]{fontenc}
\usepackage[utf8x]{inputenc}
\usepackage{ucs}
\usepackage[spanish,catalan,english]{babel}
\usepackage{graphicx}
\usepackage{subfigure}
\usepackage{listings}
\usepackage{moreverb}
\usepackage{hyperref} % option [dvipdfm] disables clickable refs
\hypersetup{pdftex, colorlinks=true, linkcolor=blue, filecolor=blue,pagecolor=blue, urlcolor=blue}

\setlength\textheight{230mm}
\setlength\textwidth{150mm}
\setlength\topmargin{+5mm}
\setlength\oddsidemargin{0mm}
\setlength\evensidemargin{0mm}
\setlength\parindent{0pt}				% no indentation on any paragraph start
\setlength\parskip{0.25\baselineskip}	% paragraphs are close
\renewcommand\baselinestretch{0.8}	% lines are closer

\renewcommand{\familydefault}{\sfdefault}    % should be similar font to auth.ps
\newcommand{\hrefx}[1]{\href{#1}{#1}} % explicit \href

% the following two commands bring \section and \subsection closer to the text
% you may have to add a line for \subsubsection if you have such
\makeatletter
\renewcommand\section{\@startsection {section}{1}{\z@} {3pt} {1pt}
{\normalfont\normalsize\bfseries}}
\renewcommand\subsection{\@startsection {subsection}{1}{\z@} {3pt}
{1pt} {\normalfont\normalsize\bfseries}}
\makeatother

% the title text needs to match auth.ps, so these changes are required
\begin{document}
\title{\Large
	\textbf{Inferno DS : Inferno port to the Nintendo DS}
}
\author{\large
	\begin{tabular}{c}
 	\textsl{Salva Peiró}\\
 	\textsl{Valencia, Spain}\\
	\texttt{saoret.one@gmail.com}\\
	\end{tabular}
}

% no page numbers in title
\maketitle\thispagestyle{empty}
% in fact, no page numbers anywhere
\pagestyle{empty}

\begin{abstract}

	\setlength\parindent{0pt}
	\setlength\parskip{0.25\baselineskip}

	This document describes the work performed in the Inferno DS port.

	It is organised as follows: 
	starts with the background and the motivation for this work,
	continues a DS hardware overview, and then discusses the development process,
	focusing on the setup and development of dis applications running on the DS.
	At the end the conclusions and future work are presented.

	\end{abstract}

\section{Background}	
The DS \cite{inferno-ds} native port of Inferno \cite{inferno-os} was started by Noah Evans for GSoC 2007 \cite{gsoc2007},
at the the end of the GSoC the port was starting to be usable under the \texttt{no\$gba} \cite{no$gba} emulator where it was possible to interact with Inferno's window manager: \textbf{wm(1)}
\footnote{
		the notation \textbf{page(section)}, refers to Inferno manual pages \cite{infernoman}
}
using the touch screen, but when running on the ``real'' DS touch wasn't working.

In spite of this all the basic functionality required to develop; like the ability to debug
using the \emph{print} statement had already been added,
also emulators are of great help to save test time.

\subsection{Motivation}
Besides sharing the same motivation stated by Noah Evans on his GSoC 2007 application \cite{gsoc2007},
there where a few more reasons to prefer a native port:
\begin{enumerate}
	\item
	After checking that \textbf{emu(1)} on DSLinux \cite{DSLinux} was not viable
	as when running with graphics the program crashed due to out of memory errors.
	There was the curiosity about the benefits of a native port, both in development and use.
	
	\item
	Overcome limitations of some homebrew programs:
	like no multi-tasking and the benefits of having a coherent system and with a standard set of tools.
	
	\item
	Have a ''real'' testbed for limbo applications,
	which could benefit applications like those developed in the inferno-lab \cite{caerwyn-ipn}.

\end{enumerate}

\section{DS Overview}
What follows is a small overview of the Nintendo DS's hardware
organized in three subsections: the system processors,
it's inter-communication mechanisms, and last the built-in devices (and expansions).

\subsection{Processors}

The DS has two 32 bits ARM \cite{arm7tdmi} processors, one more powerful ARM946E-S @ 66MHz
which is in charge of the video and performs the main computations,
and another ARM7TDMI @ 33MHz which acts as a slave to deal with the remaining devices:
wireless, audio, touch, wireless, power management, etc.

The system is shipped with the following internal memory:

\begin{itemize}
	\item 4096KB	Main ARM9 RAM
	\item 96KB	Main ARM7 WRAM (64Kb + 32K mappable to NDS7 or NDS9)
	\item 60KB	TCM/Cache (TCM: 16K Data, 32K Code) (Cache: 4K Data, 8K Code)
	\item 656KB	Video RAM (allocateable as BG/OBJ/2D/3D/Palette/Texture/WRAM memory)
	\item 256KB	Firmware FLASH (512KB in iQue variant)
	\item 36KB	BIOS ROM (4K NDS9, 16K NDS7, 16K GBA)
\end{itemize}

For more details see \cite{gbatek}[GBATEK, NDS Overview].

\subsection{Communication}

Processor communication in the DS can be performed using this methods,
together with their combinations:
\begin{itemize}
	\item Shared memory:
	The 4Mb of ARM9 RAM starting at \texttt{0x02000000}, can be shared by both processors,
	it's important to note that one of the cpus can be given priority (using the \texttt{EXMEMREG} register)
	over the other when concurrently accessing the memory.
	
	\item Hardware fifos:
	The DS fifo controller allows receiving/sending of 32 bit values from/to each cpu.
	This can be done in a full-duplex manner, where each cpu has a destination queue
	which stores the values sent by the other cpu. Notification of activity in the queues
	is performed through IRQ's to the respective cpu.

	This mechanism is crucial as it allows sending messages to request actions,
	this is used for example to read/write the RTC, obtaining the touch coordinates,
	perform wifi tasks and request audio samples to be played or recorded to the ARM7 cpu.
	
	\item Sync interrupt:
	The Sync IRQ is a simple mechanism which allows one cpu (local) to generate an IRQ
	to the other (remote) cpu, this can be use for example for emulating wifi rx activity
	as the ARM7 detects when a packet has been received and informs the ARM9 using Sync.
	
\end{itemize}

Given that accessing shared memory generates wait states to the cpu with less priority,
it must be used with care, one approach which works well is using it in combination
with fifos, by passing fifo messages with pointers to shared memory.
This is analog to the function call by value or by reference.

See \cite{gbatek}[GBATEK, DS Inter Process Communication (IPC)] for a more detailed description.

\subsection{Devices}

The system is has the following built-in devices:

\begin{itemize}
	\item Video:
	There're two LCD screens (each 256x192 pixel, 3 inch, 18bit color depth, backlight),
	each of the screens has a dedicated 2D video engine, plus one 3D video engine which
	can be assigned to each of the screens.
	
	\item Sound:
	There're 16 sound channels (16x PCM8/PCM16/IMA-ADPCM, 6x PSG-Wave, 2x PSG-Noise),
	output can be directed either to: two built-in stereo speakers, or to a headphones socket,
	while input can come either from: a built-in microphone, or microphone socket.
	
	\item Controls:
	Interacting with the DS is achieved through a gamepad and a touchscreen:
	the gamepad provides 4 direction keys plus 8 buttons,
	while the touchscreen on the lower LCD screen can be used as a pointing device.

	\item Networking:
	Wifi IEEE802.11b,  networking is provided by the RF2958 (aka RF9008) chip from RFMD.
	The main drawback of wifi, is that there is no documentation about it's  interfacing and
	 programming, instead all the code, information known has been reverse engineered.
	All the information is gathered in \cite{gbatek}[GBATEK, DS Wireless Communications]
	and also in the \textbf{dswifi} project and DSLinux \cite{DSLinux}

	\item Specials:
	Some aditional devices include: 
	Built-in Real Time Clock, Power Managment Device Hardware divide and square root functions
	and  CP15 System Control Coprocessor (cache, tcm, pu, bist, etc.)

	\item External Memory:
	There're two available slots: NDS slot (slot-1) and GBA slot (slot-2)
	which are the prefered way for plugging expansion cards and other devices.
	Their most common usage is to provide storage support to sd/tf cards.
	But there're also devices like \textbf{Dserial}, \textbf{CPLDStarter} or \textbf{Xport} \cite{xport},
	which provide uart, midi, usb and standard digital i/o interfaces together with CPLDs or FPGAs.
	
\end{itemize}

see \cite{gbatek}[GBATEK, NDS Hardware Programming].

\section{DS Port}

This section describes the idiosyncrasies of the DS port,
in particular those related with the setup, kernel and application development.

\subsection{Environment}

The development environment is the default shipped with Inferno.
The compiler used is \texttt{5\{a,c,l\}} compiler which forms part of the
\emph{Inferno and Plan 9 compiler suite} \cite{ken-cc}.
It is used to build the ARM \cite{armarm} binaries for both cpus: ARM7, ARM9,
together with the companion tools: \texttt{mk, acid, ar, nm, size, etc.} which are used for building, debugging and examining the resulting binaries.

The only special tool required is \texttt{ndstool} \cite{ndstool} which is used to generate a bootable image
to be launched by the NDS loader running on the DS, for this purpose the image contains
everything required to describe how to boot the code, this includes: 
the ARM7 and ARM9 binaries and their correspoding load addresses, entrypoints, etc.
	
\subsection{Communication: Fifos IPC}

As it's been explained the DS system is composed by two cpus,
this poses a problem when sharing the hardware devices between cpus,
to eliminate this problem the devices are assigned to one cpu or the other,
and example of this is the SPI (Serial Peripheral Interface) owned by the ARM7,
there are some (a lot, really) of devices accessed through SPI:
	touch, wifi, rtc, firmware, power management, audio, ...
The same happens with the lcd hardware which is owned by the ARM9,
as a consequence of this it's impossible to use the print statement from ARM7.

To overcome this problems  the adopted solution has been to use
the available communication mechanism: fifos and shared memory,
to provide an interface which allows communication of both cpus,
and by extension to provide access to devices not owned by the cpu.

The interface is analog to function calls, that is each message is associated
with a function which performs the work requested by the message.
For the sake of simplicity the function and it's arguments are encoded 
into a 32 bit message, where the message encoding is as follows:

\begin{center}
\begin{boxedverbatim}
	msg[32] := type[2] | subtype[4] | data[26],
	field[n] refers to a field of n bits of length
		
	type[2] := 00: System, 01: Wifi, 10: Audio, 11: reserved.
	subtype[4] := 2^4 = 16 type specific sub-messages.
	data[26] := data/parameters field of the message.
\end{boxedverbatim}
\end{center}
		
This structure has to accomodate all the required information shared by both cpus,
thus what follows is some reasoning behind the message encoding, basically it's main
purpose is to have a readable and easy to understand and manipulate notation.

\begin{description}
\item \verb+type[2]+
is used to have messages organised in 4 bit types: System, Wifi, Audio and a Reserved type.

\item \verb+subtype[4]+
is used to further cualify the message type.

For example, given message \verb+type[2] = Wifi+ there're several actions to be performed like:
\begin{itemize}
	\item initialising the wifi controller
	\item preparing for sending/receiving a packet
	\item setting the wifi authentication parameters
	\item \ldots
\end{itemize}
these can be encoded using the 16 available message \verb+subtype[4]+'s.

\item \verb+data[26]+
the lenght of the data field is not choosen 'at random', instead it's choosed
as the minimum size which can allow passing of ARM9 RAM addresses @ 0x02000000, 4Mb.
This allows passing of ''pointers'', which can hold all the required arguments.
Note: this will have to be revised when using memory expansions @ 0x08000000, 16 Mb	
\end{description}
	
\subsubsection{Fifos:  send a msg}

Here's an example extracted from devrtc.c executed by the ARM9 side to read the ARM7 RTC.

\begin{center}
\begin{boxedverbatim}
int
nbfifoput(ulong cmd, ulong data)
{
    if(FIFOREG->ctl & FifoTfull)
        return 0;
    FIFOREG->send = (data<<Fcmdlen|cmd);
    return 1;
}
...
ulong secs;
nbfifoput(F9TSystem|F9Sysrrtc, (ulong)(&secs));
\end{boxedverbatim}
\end{center}

Moreover as the fifos hardware and the implemented interface are simetric,
the same code can be used by the the ARM7 to \emph{sprint} strings and send them
to the ARM9, which will be able to output them to the LCD using \emph{print}:

\begin{center}
\begin{boxedverbatim}
int
print(char *fmt, ...)
{
    int n;
    va_list ap;
    char *sd  = SData;

    memset((void*)s, '\0', PRINTSIZE);

    va_start(ap, fmt);
    n = vsprint(s, fmt, ap);
    va_end(ap);

    while(!nbfifoput(F7print, (ulong)s));
    return n;
}
...
print("batt %d aux %d temp %d\n", IPC->batt, IPC->aux, IPC->temp);
\end{boxedverbatim}
\end{center}

\subsubsection{Fifos: receive a msg}

\begin{center}
\begin{boxedverbatim}	
static void
fifotxintr(Ureg*, void*)
{
    if(FIFOREG->ctl & FifoTfull)
        return;
    wakeup(&putr);
    intrclear(FSENDbit, 0);
}

static void
fiforxintr(Ureg*, void*)
{
    ulong v;
    while(!(FIFOREG->ctl & FifoRempty)) {
        v = FIFOREG->recv;
        fiforecv(v);
    }
    intrclear(FRECVbit, 0);
}

static void
fifoinit(void)
{
    FIFOREG->ctl = (FifoTirq|FifoRirq|Fifoenable|FifoTflush);
    intrenable(0, FSENDbit, fifotxintr, nil, "txintr");
    intrenable(0, FRECVbit, fiforxintr, nil, "rxintr");
}
\end{boxedverbatim}
\end{center}

Here \verb+fiforxintr+ is executed when an message receive IRQ is triggered,
then the fifo is examined to read the message, which is passed to \verb+fiforecv+
which knows the encoding of the messages, and invokes the corresponding
function associated with each message.

% TODO sección 'DS Port: ARM7 Kernel'
\subsection{DS kernels}

The DS port shares a lot of similarities with the other Inferno's ARM ports,
which have been used both as a source of inspiration and ideas.
In particular with the \textbf{Ipaq} port as the platform is somehow similar to the DS:
as both have touch screens, storage, audio and wireless networking,
although the underlying hardware is completely different.

Still there're certain limitations inherent to the DS that make it look a small brother:
like the 66 Mhz CPU clock, the 4 Mb of available RAM, small lcd displays and reduced wireless capabilties: only wep and open modes at 2.0 Mpbs, that should be examined once wifi code is fully working, to check it it affects the use of the styx(5) protocol to access remote filesystems.

Another interesting aspect is how the Inferno kernel running on the ARM9
provides devices like \textbf{pointer(3)}, \textbf{ether(3)}, \textbf{rtc(3)}, \textbf{audio(3)}, etc. 
which perform their work by requesting it to the ARM7 using the Fifo IPC seen above.

\subsubsection{ARM7}
While the ARM9 cpu has 4 Mb of RAM, which permits it to run an Inferno kernel,
the smaller ARM7 has only access to 64 Kb or EWRAM (exclusive RAM).

Given this memory limitation the ARM7 can't run a Inferno kernel,
but as it's been discussed above the ARM7 is required in order to access
the devices owned by the ARM7.
For this purpose runs specific code which basically manages the needed hardware devices
and provides the fifo interface comented above in order the ARM9 can use it.

\subsection{Application}
% TODO sección 'DS Port: Applications'

At the application level the DS has some features that make it interesting:
	
The input interface: buttons and touchpad with the graphical output:
two small lcd displays which present a challenge to developing applications for it.

This has implications in the graphical user interface to use,
which is being object of experimentation in the inferno-lab \cite{caerwyn-ipn},
to find out how to best use the available lcd screens with the lower touch screen.

This opens field for interesting applications, which combine
graphics, touch, networking and audio.
This can include games, VoIP, music, midi synths toghether with
other common uses, like: connecting/managing remote systems with \textbf{cpu(1)},
or accessing to remote resources using the \textbf{styx(5)} protocol.
		
\subsubsection{Development}

With a standard Inferno distribution placed on a sd/tf card,
it's seamlessly to setup Inferno running on the Nintendo DS,
as it only requires an Inferno kernel which can be distributed as .nds image
available for download from the Inferno DS project site \cite{inferno-ds}.
This kernel can be transfered to a sd/tf card, to be booted by the NDS loader.

This kernel provides access to the underlying hardware through devices \textbf{dev(10)},
this is: through a filesystem interface which is used by the applications to
make use of the kernel services: \textbf{draw(3)}, \textbf{pointer(3)}, \textbf{ether(3)}, \textbf{audio(3)}
and a specific \textbf{devdldi} which provides storage access to sd/tf cards.

With all this, the development of applications consists of the following steps:
\begin{enumerate}
	\item setup Inferno emu on a development host:
	where the applications can be coded, compiled and tested, see \cite{ipwl} for more details.
	
	\item test applications on a DS emulator \emph{(optional)}:
	like no\$gba \cite{no$gba} or desmume.

	\item transfer applications (.dis files) to sd/tf card:
	to be launched after booting the Inferno DS kernel.
\end{enumerate}

\section{Conclusions}
% TODO

The main conclusion extracted during the development of the port has been
how the careful design and implementation of the whole Inferno system
have made the task of developing this port easier.

This has had also an effect on the tasks of locating and fixing errors,
and introducing new functionality like input, storage, networking and audio
which have become easier.

The benefits of the Inferno design \cite{inferno-os} will be also noticed
when developing limbo applications for the ds,
as this area has been less used/tested during the development of the port.

\section{Future work}
As there're always things to do or rework
this can be regarded as a \emph{work in progress},
in particular the graphics side and audio are being reworked
one to take advantage of both lcd screens under Inferno,
and the other to improve audio playing and recording quality.

Another area that needs atention is the wireless networking,
whose code needs to be tested and finished,
as this will open field by communicating it with other devices,
this will allow to boot remote kernels, access filesystems provided by a \textbf{emu(1)} instance running hosted, etc.
	
As things are being polished the work to be done will move from the kernel side to the applications side,
as it's already happening with the inferno-lab \cite{caerwyn-ipn} experiments with the Mux interface and
with the quong/hexinput keyboard to ease the interaction with the system.
	
\newpage

\begin{thebibliography}{1}
\selectlanguage{english}

\bibitem[1]{inferno-ds}
	Noah Evans, Salva Peiró, Mechiel Lukkien
	\newblock {``Inferno DS: Native Inferno Kernel for the Nintendo DS''}.
	\hrefx{http://code.google.com/p/inferno-ds/}.

\bibitem[2]{inferno-os}
	Sean Dorward, Rob Pike, David Leo Presotto, Dennis M. Ritchie, Howard Trickey, Phil Winterbottom
	\newblock {``The Inferno Operating System''}.
	\newblock{Computing Science Research Center, Lucent Technologies, Bell Labs, Murray Hill, New Jersey USA}
	\hrefx{http://www.vitanuova.com/inferno}.
	\hrefx{http://code.google.com/p/inferno-os/}.

\bibitem[3]{gsoc2007}
	Noah Evans, mentored by Charles Forsyth,
	\newblock {``Inferno Port to the Nintendo DS''}.
	\newblock Google Summer of Code 2007,
	\newblock \hrefx{http://code.google.com/soc/2007/p9/about.html}.

\bibitem[4]{no$gba}
	Martin Korth,
	\newblock {``no\$gba emulator debugger version''}.
	\hrefx{http://nocash.emubase.de/gba-dev.htm}.

\bibitem[5]{DSLinux}
	Pepsiman, Amadeus and others,
	\newblock {``DSLinux: port of uCLinux to the Nintendo DS''}.
	\hrefx{http://www.dslinux.org}.
	
\bibitem[6]{caerwyn-ipn}
	Caerwyn Jones \& co,
	\newblock {``Inferno Programmers Notebook''}.
	\newblock \hrefx{http://caerwyn.com/ipn}, \hrefx{http://code.google.com/p/inferno-lab}

\bibitem[7]{arm7tdmi}
	ARM (Advanced Risc Machines),
	\newblock {``ARM7TDMI (rev r4p3) Technical Reference Manual''}.
	\newblock ARM Limited,
	\newblock \hrefx{http://www.arm.com/documentation/ARMProcessorCores}.
	
\bibitem[8]{gbatek}
	Martin Korth,
	\newblock {''GBATEK: Gameboy Advance / Nintendo DS Technical Info''}.
	\hrefx{http://nocash.emubase.de/gbatek.txt}.
	\hrefx{http://nocash.emubase.de/gbatek.htm}.

\bibitem[9]{xport}
	Charmed Labs,
	\newblock {''Xport''}.
	\hrefx{http://www.drunkencoders.org/reviews.php}.

\bibitem[10]{ndstool}
	DarkFader, natrium42, WinterMute, 
	\newblock {``ndstool Devkitpro: toolchains for homebrew game development''}.
	\newblock \hrefx{http://www.devkitpro.org/}
	
\bibitem[11]{ken-cc}
	Ken Thompson,
	\newblock {``Plan 9 C Compilers''}.
	\newblock Bell Laboratories, Murray Hill, New Jersey 07974, USA.
	\newblock \hrefx{http://plan9.bell-labs.com/sys/doc/compiler.html}.

\bibitem[12]{armarm}
	David Seal,
	\newblock {``The ARM Architecture Reference Manual'', 2nd edition}.
	\newblock Addison-Wesley Longman Publishing Co.
	\newblock \hrefx{http://www.arm.com/documentation/books.html}.
	
\bibitem[13]{ipwl}
	Phillip Stanley-Marbell,
	\newblock {``Inferno Programming with Limbo''}.
	\newblock John Wiley \& Sons 2003,
	\newblock \hrefx{http://www.gemusehaken.org/ipwl/}.

\bibitem[14]{infernoman}
	\newblock {``The Inferno Manual''}.
	\newblock \hrefx{http://www.vitanuova.com/inferno/man/}.
	
\end{thebibliography}

\end{document}
